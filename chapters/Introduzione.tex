\section{Introduzione}
\subsection{Scenario}

Nell'ambiente in cui viviamo siamo circondati da sempre più entità computazionali eterogenee tra loro. Le specifiche di basso livello, come efficienza, organizzazione e coordinazione di questi dispositivi in un sistema distribuito, influenzano pesantemente le scelte di design di quest'ultimo. Sorge quindi la necessità di un nuovo paradigma di sviluppo, che consenta di astrarre i dettagli del sistema e permetta di definirne il comportamento collettivo. Una risposta a questo problema è fornita dal paradigma di \textit{aggregate computing}, che introduce il concetto di \textit{field computazionale}: una ``mappa'' dispositivo-valore variabile nel tempo. Protelis è un linguaggio di programmazione che offre queste caratteristiche.
% TODO: Da finire
% * building blocks
% * obiettivo

% Cosa esiste, perché si fa
% Esiste Protelis, utilizzato in Alchemist, non esiste alcuna applicazione reale che lo utilizza

\subsection{Obiettivo}

L'obiettivo della presente tesi è di fornire una serie di implementazioni del \textit{backend} di Protelis che possano essere utilizzate come riferimento per l'utilizzo industriale del linguaggio.
