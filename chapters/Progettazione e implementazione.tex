\subsection{Descrizione degli internals di Protelis}

\subsubsection{ProtelisVM}

ProtelisVM è la parte centrale del linguaggio: consente l'esecuzione di un programma in un device.

\subsubsection{ExecutionContext}

ExecutionContext è lo strato che si pone tra la virtual machine e l'ambiente nel quale questa viene eseguita. È responsabile di tre compiti:

\begin{itemize}
    \item{mantenere lo stato della virtual machine locale attraverso i cicli di esecuzione;}

    \item{mantenere lo stato condiviso dai dispositivi vicini;}

    \item{interfacciarsi con l'ambiente nel quale il dispositivo è \textit{embedded}, come tempo, spazio o sensori;}
\end{itemize}

\subsubsection{AbstractExecutionContext}

AbstractExecutionContext è una realizzazione parziale di ExecutionContext: implementa le tutte le funzioni che dovrebbero essere fornite da un \textit{context}. In particolare utilizza un NetworkManager per la funzione di condivisione dello stato con i dispositivi vicini.

\subsubsection{NetworkManager}

Il NetworkManager è l'astrazione di rete utilizzata dalla virtual machine. È il componente che si deve occupare dello scambio di messaggi riguardanti lo stato con gli altri dispositivi del sistema.

\subsection{Il mio contributo}

\subsubsection{Device}

Qui sotto dovrei avere i miei UML.

\begin{center}
\begin{plantuml}
@startuml

Interface NetworkManager

Device -- ProtelisProgram : executes
Device -- NetworkManager : uses
Device -- DeviceCapabilities : features

class Device {
  ProtelisProgram program
  NetworkMananger nm

  runCycle()
}

class ProtelisProgram {

}
@enduml
\end{plantuml}

\end{center}

Ma pure qua sopra.

\subsection{Esempio memoria condivisa}

\subsection{Esempio con socket}

\subsection{Esempio con protocollo MQTT}
