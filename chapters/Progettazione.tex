\section{Progettazione}

Si vuole modellare una rete di dispositivi che siano in grado di comunicare tra
di loro. Viene introdotto l'elemento \texttt{Device}: un generico dispositivo in una rete
in grado di eseguire programmi Protelis e di comunicare con altri dispositivi
vicini. Ciascun dispositivo deve integrare al proprio interno una macchina
virtuale Protelis in modo da poter eseguire un dato programma.

Per semplicità si assume che i dispositivi abbiano già eseguito la fase di
discover e che quindi ciascun dispositivo conosca i propri vicini.

\subsection{Wrapping delle API}

\subsubsection{Device}
Un \texttt{Device} rappresenta un generico dispositivo in grado di eseguire programmi
Protelis, per questo scopo contiene al proprio interno una \texttt{ProtelisVM}. Il dispositivo si trova all'interno di una rete e deve essere in grado
di comunicare con i propri vicini per conoscere gli aggiornamenti del loro
stato.

\begin{center}
  \begin{plantuml}
    @startuml
    Device -- ProtelisProgram : executes
    Device -- NetworkManager : uses
    Device -- DeviceCapabilities : features

    Interface NetworkManager
    class Device {
      ProtelisProgram program
      NetworkMananger nm

      void runCycle()
    }

    class ProtelisProgram { }
    @enduml
  \end{plantuml}
\end{center}

Ogni device è contraddistinto da un \texttt{DeviceUID} univoco, che in questo
progetto è stato realizzato con un numero intero incrementale che parte da
0. Nel metodo costruttore gli viene passata l'istanza di
\texttt{ProtelisProgram} che dovrà eseguire. Al proprio interno instanzia
un'opportuna implementazione di \texttt{ExecutionContext}, il quale riferimento
viene passato alla macchina virtuale. Espone un metodo \texttt{runCycle()} per
eseguire in maniera sincrona un ciclo di esecuzione della macchina virtuale al
proprio interno. Utilizza lo \textit{strategy pattern} per definire qual'è la
modalità utilizzata dal dispositivo per comunicare con gli altri nodi. Ogni
singola implementazione di \texttt{NetworkManager} verrà, in seguito, discussa
singolarmente. Allo stesso modo utilizza uno \texttt{Speaker}, ovvero
un'implementazione del metodo \texttt{announce()} utilizzato dal programma
Protelis che verrà fatto eseguire all'interno del dispositivo.



\subsubsection{DeviceCapabilities}
Realizza le tre caratteristiche richieste
dall'\texttt{ExecutionContext} (\ref{sec:ExecutionContext}). Implementa anche
l'interfaccia \texttt{Speaker} e realizza il metodo \texttt{announce()}
utilizzato all'interno del programma Protelis che verrà eseguito.

\begin{center}
  \begin{plantuml}
    @startuml

    AbstractExecutionContext <|- DeviceCapabilities
    Speaker -- DeviceCapabilities
    NetworkManager -- DeviceCapabilities

    abstract class AbstractExecutionContext
    interface Speaker
    class DeviceCapabilities {
      void announce()

    }
    @enduml
  \end{plantuml}
\end{center}
