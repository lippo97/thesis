\message{ !name(chapter-two.tex)}
\message{ !name(chapter-two.tex) !offset(74) }
\section{Architettura riusabile}
L'obiettivo di questa sezione è di modellare una architettura del backend di
Protelis abbastanza flessibile, che consenta di incorporare la macchina virtuale
all'interno di un dispositivo, così che attraverso questo sia possibile
eseguirne le iterazioni. Concretamente si effettuerà un mapping delle
caratteristiche di un dispositivo (livello in basso nella figura
\ref{fig:abstraction-layers}), così che queste possano essere utilizzate per
implementare i costrutti del \textit{field calculus}. Un obiettivo importante di
questo modello sarà circoscrivere in un'entità ben definita il concetto di
strategia di comunicazione. In questo modo il device sarà in grado di funzionare
indipendentemente dal metodo specifico scelto per realizzarla.

% TODO: da scrivere bene un'introduzione
\subsection{API di Protelis}
Le API del backend di Protelis sono scritte in Java, quindi sono facilmente
integrabili anche con linguaggi eseguiti sulla Java Virtual Machine. Esse
offrono le seguenti astrazioni.

\subsubsection{ExecutionContext}
Interfaccia che si pone fra una macchina virtuale Protelis e l'ambiente da cui
essa è circondata. I suoi compiti sono tre:
\begin{itemize}
\item tenere traccia dello stato persistente attraverso le iterazioni successive
  del prorgamma;
\item tenere traccia dell'ultimo stato condiviso dai dispositivi vicini;
\item tenere traccia dell'interazione del dispositivo con l'ambiente esterno
  (tempo, spazio, sensori, attuatori, ecc).
\end{itemize}

\subsubsection{ProtelisVM}
La macchina virtuale Protelis è il nucleo centrale dell'architettura: contiene
l'interprete del linguaggio che al proprio interno implementa gli operatori del
\textit{field calculus}. Accetta in input un \texttt{ProtelisProgram} e utilizza un
\texttt{ExecutionContext} per mantenere il proprio stato e interfacciarsi con
l'esterno. L'interfaccia permette di azionarne i cicli di esecuzione.

\begin{figure}
  \centering
  \includegraphics[width=\linewidth]{diagrams/output/protelis-api}
  \caption{Relazione tra \texttt{ProtelisVM}, \texttt{ExecutionContext} e
      \texttt{ProtelisProgram}.}
  \label{fig:uml-protelisvm}
\end{figure}

\subsubsection{AbstractExecutionContext}
L'interfaccia \texttt{ExecutionContext} contiene la definizione di molti metodi
che dovrebbero essere comuni a tutte le implementazioni di Protelis. Questa
entità astratta realizza quelle procedure delegando a una nuova entità, il
\texttt{NetworkManager}, il compito di stabilire un metodo efficace di
comunicazione.

\subsubsection{NetworkManager}
\label{sec:network-manager}
Astrazione di rete utilizzata dalla macchina virtuale Protelis: ad ogni
iterazione, la macchina virtuale ha bisogno di accedere all'ultimo stato ricevuto
dai vicini e di aggiornare lo stato che verrà inviato ad essi.

Una considerazione importante da fare è che la documentazione ufficiale
specifica che non è necessario che lo stato venga ricevuto o inviato ad ogni
iterazione. La scelta della frequenza di aggiornamento dello stato è demandata
alla specifica implementazione di questa interfaccia, per trovare il rapporto
giusto tra efficienza e condivisione dello stato.

\begin{figure}
  \centering
  % \def\svgscale{2}
  \includegraphics[width=\linewidth]{diagrams/output/networkmanager}
    \caption{Introduzione di \texttt{AbstractExecutionContext} e \texttt{NetworkManager}.}
  \label{fig:uml-networkmanager}
\end{figure}

\subsubsection{CodePath}
Rappresenta un percorso, che parte dalla radice dell'albero di esecuzione della
macchina virtuale Protelis e termina in uno dei nodi. Supportando la
serializzazione, viene utilizzato per confrontare l'esecuzione locale a un nodo
con quella dei propri vicini e permettere l'allineamento del codice. 

La specifica implementazione di questa classe è critica per la dimensione dei
pacchetti generati dalla comunicazione tra i nodi. Sarà necessario, quindi,
porre l'attenzione su questo aspetto quando la comunicazione sarà basata su
protocolli di rete.
\subsection{Definizione di nuovi concetti}
I concetti che seguono non fanno parte delle API di Protelis, ma sono stati
definiti appositamente per estenderle e creare quindi un nuovo modello, che sia
adatto a rappresentare dispositivi sia in un ambiente reale che in uno simulato.
Questo modello rappresenta dunque una possibile base di partenza per la
realizzazione di un backend di Protelis.

\subsubsection{Capacità di un dispositivo}
\label{sec:device-capabilities}

\begin{figure}
  \centering
  \includegraphics[width=\linewidth]{diagrams/output/device}
  \caption{Introduzione di \texttt{DeviceCapabilities} e \texttt{Device}.}
  \label{fig:uml-device}
\end{figure}

L'obiettivo di questa entità è di realizzare un \texttt{ExecutionContext},
traendo vantaggio delle funzioni già implementate in
\texttt{AbstractExecutionContext}. Implementa le seguenti caratteristiche di un dispositivo:
\begin{itemize}
\item \textbf{Mantenere uno stato nel tempo}: ovvero essere in grado di
  mantenere un insieme di variabili locali e aggiornarle in caso di necessità.

\message{ !name(chapter-two.tex) !offset(-108) }
