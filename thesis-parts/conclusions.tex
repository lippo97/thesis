\chapter{Conclusioni e lavori futuri}
\section{Conclusioni}
È stato introdotta la programmazione aggregata come una possibile soluzione a
numerosi problemi relativi ai sistemi distribuiti moderni. In particolare si è
posta l'attenzione su come il \textit{field calculus} possa formalizzare, quindi
rendere analizzabili questi problemi. Quindi è stato proposto Protelis, come
implementazione del \textit{field calculus}, che per la sua toolchain, le sue
caratteristiche e la sua portabilità rappresenta un linguaggio protagonista in
questo settore.

Partendo dalle API esistenti di Protelis, è stato definito un modello riusabile
che consente di eseguire una macchina virtuale Protelis su diverse
infrastrutture, che possono essere reti reali, come nell'Internet-of-Things
oppure simulate (come nel caso di Alchemist).  Per fare ciò si è dovuto
introdurre il concetto di middleware, uno strato che si occupa della
comunicazione tra i nodi in maniera completamente trasparente ad essi, ovvero in
modo che se l'implementazione di questo cambia, il comportamento finale del
sistema rimane lo stesso. Quindi per isolare il concetto di comunicazione tra
dispositivi dalle entità esistenti, si è fatto uso del concetto, già definito
dalle API di Protelis, \texttt{NetworkManager}, che fornisce una maniera molto
semplice di realizzare lo scambio di messaggi.

A sostegno di questo modello, sono stati portati tre diversi esempi che
rappresentano tre diverse realtà in cui il linguaggio Protelis può essere
utilizzato. In particolare il primo mira a creare un ambiente simulato, dove
viene eseguita una simulazione di una rete reale i cui nodi eseguono una
macchina virtuale Protelis; il secondo mostra come l'architettura di Protelis
possa essere facilmente estesa ad un sistema distribuito, supportando la
comunicazione attraverso la rete IP; infine il terzo implementa l'uso di uno dei
protocolli più utilizzati in ambito Internet-of-Things, settore in cui il
linguaggio Protelis è sicuramente competitivo.

Gli elaborati prodotti sono stati integrati come esempi al repository ufficiale
di Protelis\footnote{https://github.com/Protelis/Protelis-Demo}. Rappresentano
il punto di partenza per chiunque voglia implementare un nuovo backend del
linguaggio.

\section{Lavori futuri}
Gli elaborati proposti in questa tesi sono facilmente estendibili. Di seguito
vengono elencati alcuni suggerimenti, per evolvere le demo esistenti o per
svilupparne di nuove, che potranno essere presi in considerazione.  Migliorare
la demo è molto facile: è sufficiente fare un
\textit{fork}\footnote{https://help.github.com/en/github/getting-started-with-github/fork-a-repo}
del repository ufficiale della demo di Protelis, effettuare delle modifiche e
proporle al repository originale tramite una
\textit{pull-request}\footnote{https://help.github.com/en/github/collaborating-with-issues-and-pull-requests/about-pull-requests}.

\subsection{Introduzione di nuove capacità}
Il programma eseguito dai nodi in questi esempi è un semplice Hello, World!
seguito da un conteggio a ritroso. Questo perché la sua finalità non è mostrare
le capacità di Protelis di sfruttare le capacità di un dispositivo reale, bensì
di mostrare come esso si possa adattare a tecniche di comunicazione diverse.  Al
fine di ampliare le dimostrazioni esistenti si potrebbe integrare alle capacità
di un dispositivo: la facoltà di interagire con lo spazio che lo circonda, la
possibilità di interagire con il tempo. In questo modo ciascun dispositivo
acquisirebbe la consapevolezza della propria posizione nello spazio-tempo,
garantendo la possibilità di fruire di nuove tipologie di algoritmi, come il
gradiente. Una possibile e affascinante applicazione di questa nuova tipologia
di dispositivi è il controllo di uno sciame di droni. Infatti, garantendo
ciascuno il controllo della propria posizione nel tempo, ma descrivendone il
comportamento in maniera collettiva, è relativamente semplice prevenire
qualsiasi tipo di collisione, fornendo la possibilità di creare comportamenti o
coreografie in maniera efficace.  Protelis fornisce supporto per l'impiego delle
funzioni relative allo spazio e al tempo tramite le interfacce
\texttt{SpatiallyEmbeddedDevice} e \texttt{TimeAwareDevice}.

\subsection{Migliorare isolamento}
Nell'architettura proposta nel corso di questa tesi, alcune classi possono
essere difficilmente estendibili, per esempio se si volesse estendere la classe
\texttt{DeviceCapabilities} di un \texttt{Device} per implementare
\texttt{TimeAwareDevice}, un oggetto generico capace di lavorare con le unità
temporali, si troverebbero difficoltà a farlo in maniera semplice, ma sarebbe
necessario estendere \texttt{DeviceCapabilities}. Se volessimo a quel punto
implementare \texttt{SpatiallyEmbeddedDevice}, un oggetto consapevole della
propria posizione nello spazio, sulle stesse capacità dell'oggetto, ci
troveremmo di nuovo a dover estendere la classe esistente. Un altro caso
interessante è l'impossibilità di utilizzare uno \texttt{DeviceUID} che non sia
implementato tramite un numero intero.  Potrebbe essere quindi eseguita
un'operazione di refactoring di tutta la codebase, con la finalità di
estrapolare e isolare questi comportamenti in una sola classe, come già
effettuato con la classe \texttt{NetworkManager}, in modo da poterne utilizzare
di diversi in base alle richieste specifiche di un problema. Il pattern strategy
si presta molto bene a questo tipo di esigenze.

\subsection{Ulteriori template}
Si è visto che l'architettura prodotta, per mezzo dell'entità
\texttt{NetworkManager}, si presta all'utilizzo di un numero vastissimo di
tecniche e protocolli di comunicazione esistenti nel panorama odierno. Un altro
possibile miglioramento può essere l'implementazione di nuovi
\texttt{NetworkManager}, che possano utilizzare tecniche diverse da quelle già
utilizzate. In questo modo il repository ufficiale verrebbe arricchito di nuovi
esempi, che possono essere un punto di riferimento per chi si approccia al
linguaggio. Alcune possibili tecniche di comunicazione che possono essere
utilizzate sono:
\begin{itemize}
\item \textbf{STOMP} (Simple Text-orientated Messaging Protocol): protocollo che
  mira ad offrire un canale di comunicazione basato su messaggi di testo, molto
  leggibile per l'uomo, meno efficiente in termini di banda.
\item \textbf{COaP} (Constrained Application Protocol): protocollo sviluppato
  per l'interazione tra macchine. Si basa su un model REST, esattamente come
  HTTP, per offrire i propri servizi. È molto competitivo in ambito
  Internet-of-Things a causa della sua capacità di essere eseguito anche in
  dispositivi con pochissime risorse.
\item \textbf{HTTP} (HyperText Transfer Protocol): protocollo destinato allo
  scambio di informazioni, come ipertesto, risorse in file di testo o streaming
  video. Proprio a causa della sua flessibilità potrebbe essere un possibile
  candidato per una versione, probabilmente non competitiva, ma funzionante e
  semplice da implementare di questo progetto.
\end{itemize}

\subsection{Protelis su dispositivi mobili}
Protelis si basa sull'infrastruttura esistente di Java, infatti viene eseguito
all'interno di una Java Virtual Machine. Questa sua caratteristica può essere
sfruttata per l'implementazione di una versione di Protelis in grado di essere
eseguita all'interno di un dispositivo Android. Infatti le applicazioni per
questi dispositivi sono scritte nativamente in Java e Kotlin, due linguaggi con
cui le API di Protelis si possono interfacciare molto semplicemente. Inoltre la
grandissima quantità di sensori presenti in un dispositivo mobile, come uno
smartphone, allarga notevolmente il panorama delle possibili applicazioni che la
programmazione aggregata potrebbe consentire. Ulteriore vantaggio dell'utilizzo
in dispositivi mobili è la facilità di implementare la comunicazione. Basti
pensare che ogni smartphone moderno possiede un'antenna Bluetooth e una per la
ricezione Wi-Fi. Esempi di possibili applicazioni sono quelli menzionati alla
Sezione \ref{sec:aggregate-computing}.

% \subsection{Ubiquitous computing}
% Ubiquitous computing o ubicomp è un concetto dell'ingegneria del software che si
% contrappone al tipico uso desktop del computer: mentre con questo l'utente deve
% cercare un interazione con un dispositivo come un personal computer, uno
% smartphone, un tablet o un oggetto smart generico, nell'ubicomp l'utente è
% circondato dalla computazione, che avviene in ogni momento e ovunque. Essa può
% avvenire, per esempio, all'interno di un paio di occhiali. La finalità di questa
% è di supportare l'utente nelle sue scelte, nei suoi compiti quotidiani, senza
% che questo lo richieda direttamente. L'Internet-of-Things è una specializzazione
% di questo campo in cui l'interazione è focalizzata tra dispositivi.
%
% La programmazione aggregata, che fornisce costrutti affidabili e robusti per
% realizzare in maniera semplice applicazioni distribuite, è una delle possibili
% tecniche per realizzare questo tipo di sistemi.

% supporto per sistemi ibridi reale-simulato ubicomp
